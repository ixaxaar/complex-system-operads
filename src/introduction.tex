% Introduction section
\lettrine[lines=2, findent=3pt, nindent=0pt]{C}{omplex systems} are characterized by numerous interacting components that exhibit emergent behaviors and phase transitions not evident from the study of individual components \citep{mitchell2009complexity}. Phase transitions in complex systems are marked by the emergence of sudden, qualitative changes where new collective properties arise from interconnected components. Understanding these transitions requires mathematical frameworks that can capture the multi-scale, hierarchical nature of complex systems.

\subsection{Key Properties of Complex Systems}

Complex systems exhibit two fundamental structural properties: near-decomposability and scale-invariance. \textit{Near-decomposability}, first identified by \citet{simon1962architecture}, refers to systems that can be broken down into subsystems that interact weakly with each other while maintaining strong internal interactions. This property creates hierarchical organizations where subsystems are nested within larger systems. For instance, cells form tissues, which form organs, which form organisms—each level having its own internal dynamics while participating in broader systemic functions.

This hierarchical organization facilitates the emergence of new properties at different levels of abstraction, as local interactions lead to global phenomena. In biological systems, molecular interactions give rise to cellular behaviors, which in turn produce emergent patterns at the tissue or organism level \citep{battiston2020networks}. Similarly, in social systems, individual behaviors aggregate into collective phenomena such as consensus formation or cultural patterns \citep{castellano2009statistical}.

Complementing this property, \textit{scale-invariance} describes how patterns exhibited across different scales of hierarchical organization are often self-similar, manifesting in power-law distributions and fractal-like structures \citep{west2017scale}. Scale-invariant systems lack a characteristic scale—similar patterns repeat whether observed at micro or macro levels. Together, these properties create systems where local changes in compositional structure can cascade across scales to produce system-wide transformations—the essence of phase transitions in complex systems.

\subsection{Evolution of Mathematical Models}

The challenge of modeling complex systems has led to an evolution of increasingly sophisticated mathematical frameworks, each addressing limitations of its predecessors:

\begin{itemize}[leftmargin=*]
  \item \textbf{Network models} capture pairwise interactions between system components, providing a first approximation but missing crucial higher-order effects.

  \item \textbf{Hypergraphs} extend networks to include higher-order interactions, allowing for the representation of multi-agent interactions and enabling the analysis of systems with more intricate connection patterns \citep{benson2016higher, battiston2020networks}.

  \item \textbf{Simplicial complexes} further extend these models to include higher-dimensional interactions, capturing simultaneous multi-agent interactions and enabling topological analysis \citep{battiston2020networks, petri2014homological}.
\end{itemize}

\subsubsection{Networks}

Network models have been widely deployed across diverse domains, from social systems to biological networks to technological infrastructures \citep{boccaletti2006complex, newman2003structure}. By representing entities as nodes and their interactions as edges, networks reveal system properties such as connectivity patterns, robustness, and information flow \citep{barabasi2004network}. These models have successfully captured various dynamics, including synchronization in neural networks \citep{arenas2008synchronization}, opinion formation in social networks \citep{castellano2009statistical}, cascading failures in infrastructure \citep{watts2002simple}, percolation phenomena \citep{cohen2010complex}, epidemic spreading \citep{pastor2015epidemic}, and power grid failures \citep{dobson2007complex}.

\subsubsection{Hypergraphs}

Hypergraphs generalize networks by introducing hyperedges that can connect multiple nodes simultaneously \citep{berge1984hypergraphs, battiston2020networks}. Unlike standard networks where edges connect exactly two nodes, hyperedges map sets of source nodes to sets of target nodes. This structure better represents multi-agent interactions in systems like social networks \citep{zhou2007learning}, biochemical pathways \citep{klamt2009hypergraphs}, and technological infrastructures \citep{xu2013hypernetwork}. Hypergraphs have successfully modeled complex phenomena including social consensus formation across multiple influence layers \citep{neuhäuser2021consensus}, protein complex assembly and disassembly \citep{ramadan2020hypergraph}, financial contagion with directional interactions \citep{hüser2020financial}, and coordinated neural firing patterns \citep{petri2014homological}.

\subsubsection{Simplicial Complexes}

Simplicial complexes offer an even richer generalization by representing multi-dimensional interactions as geometric objects \citep{petri2014homological, giusti2016two, sizemore2018importance}. A simplicial complex consists of simplices of varying dimensions—points (0-simplices), edges (1-simplices), filled triangles (2-simplices), solid tetrahedra (3-simplices), and higher-dimensional analogues—that capture interactions between corresponding numbers of nodes. This representation enables sophisticated topological analysis of complex systems, revealing structural features not visible through network or hypergraph representations. Researchers have successfully applied simplicial complexes to analyze brain functional networks \citep{petri2014homological}, neural coding schemes \citep{giusti2016two}, and brain development \citep{sizemore2018importance}.

\subsection{Limitations}

Despite their increasing sophistication, these models still struggle to adequately represent systems with complex hierarchical organization and nested compositional structures. Specifically, they face limitations when modeling:

\begin{itemize}[leftmargin=*]
  \item \textbf{Social consensus formation}, where opinions cascade through multiple hierarchical levels of influence \citep{watts1998collective}
  \item \textbf{Biological signaling networks}, where protein complexes dynamically assemble and disassemble to form higher-order functional units \citep{strogatz2001exploring, giovannoni2017dynamic}
  \item \textbf{Financial contagion}, where institutional interactions have directionality, composition changes, and cross-scale effects \citep{farmer2009economy}
  \item \textbf{Neural synchronization}, where firing patterns coordinate across hierarchical scales \citep{bar2008dynamics, linde2021operad}
\end{itemize}

These limitations stem from the inability of existing models to fully capture the compositional nature of complex systems, where higher-level structures emerge from the composition of lower-level components in ways that preserve certain properties while giving rise to new ones.

This paper presents a novel framework that progresses beyond networks and simplicial complexes to operadic models for analyzing phase transitions and emergent phenomena.
