% Introduction section
\lettrine[lines=2, findent=3pt, nindent=0pt]{C}{omplex systems} are characterized by numerous interacting components that exhibit emergent behaviors and phase transitions not evident from the study of individual components \citep{mitchell2009complexity}. Phase transitions in complex systems are marked by the emergence of sudden, qualitative changes where new collective properties arise from interconnected components.

\subsection{Near-Decomposability and Scale-Invariance}

Two fundamental properties that characterize complex systems are near-decomposability and scale-invariance. Near-decomposable systems, as identified by \citet{simon1962architecture}, can be broken down into subsystems that interact weakly with each other while maintaining strong internal interactions. Complex systems are often organized hierarchically, with subsystems nested within larger systems. This hierarchical organization allows for the emergence of new properties at different levels of abstraction, as local interactions can lead to global phenomena. For example, in biological systems, cellular processes can interact to produce emergent behaviors at the tissue or organism level \citep{battiston2020networks}. Similarly, in social systems, individual behaviors can lead to collective phenomena such as consensus formation or opinion dynamics \citep{castellano2009statistical}. The patterns exhibited across differenet scales of heirarchical  organization are often self-similar, manifesting in power-law distributions and fractal-like structures \citep{west2017scale}. These properties are particularly relevant to understanding phase transitions, as they provide insights into how local changes in compositional structure can cascade across scales to produce system-wide transformations.

\subsection{Network Model Families}

The network-like families of mathematical structures that have been used to model complex systems can be broadly categorized into three types:

\begin{itemize}[leftmargin=*]
  \item \textbf{Network models} capture only pairwise interactions, providing a first approximation but missing crucial higher-order effects.

  \item \textbf{Hypergraphs} extend networks to include higher-order interactions, allowing for the representation of multi-agent interactions and enabling the analysis of complex systems with more intricate structures \citep{benson2016higher, battiston2020networks}.

  \item \textbf{Simplicial complexes} extend networks to include higher-dimensional interactions, capturing simultaneous multi-agent interactions and enabling topological analysis \citep{battiston2020networks, petri2014homological}.
\end{itemize}

\subsubsection{Networks}

Network models have been widely used to represent complex systems across diverse domains, from social networks to biological systems to technological infrastructures \citep{boccaletti2006complex, newman2003structure}. A network consists of nodes (vertices) connected by edges (links), representing pairwise interactions between entities. The structure of a network can reveal important properties of the system, such as its connectivity, robustness, and information flow \citep{barabasi2004network}. Networks have also been used to capture dynamics of complex systems, including synchronization e.g. in neural networks \citep{arenas2008synchronization}, opinion formation in social networks \citep{castellano2009statistical}, and cascading failures in infrastructure networks \citep{watts2002simple}, percolation theory \citep{cohen2010complex}, and epidemic spreading \citep{pastor2015epidemic}, cascading failures such as in power grids \citep{dobson2007complex}, and many other phenomena.

\subsubsection{Hypergraphs}

Hypergraphs are a generalized model of networks which contain hyperedges \citep{berge1984hypergraphs, battiston2020networks}. Hyperedges map a set of source nodes to a set of target nodes, where each edge can span multiple sources and/or targets. Hypergraphs can capture higher-order interactions that are not easily represented in traditional networks \citep{benson2016higher}. They have been used to model complex systems with multi-agent interactions, such as social networks \citep{zhou2007learning}, biological systems \citep{klamt2009hypergraphs}, and technological infrastructures \citep{xu2013hypernetwork}. Examples of such interactions include social consensus formation where collective opinions cascade through multiple layers of influence \citep{neuhäuser2021consensus}, biological signaling networks where protein complexes assemble and disassemble dynamically \citep{ramadan2020hypergraph}, financial contagion where interactions between institutions have directionality and changing compositions \citep{hüser2020financial}, and neural synchronization where coordinated firing patterns involve cascades across hierarchical scales \citep{petri2014homological, giusti2016two, sizemore2018importance}.

\subsubsection{Simplicial Complexes}

Simplicial complexes are a further generalization of networks that include higher-dimensional interactions \citep{petri2014homological, giusti2016two, sizemore2018importance}. A simplicial complex is a collection of 1, 2, 3 ... n higher-dimensional simplices that capture interactions between 1, 2, 3 .. n nodes. For example, a 0-simplex is a node, a 1-simplex is an edge, a 2-simplex is a filled triangle, and a 3-simplex is a solid tetrahedron.  Simplicial complexes can capture complex interactions among multiple entities, such as three-way or four-way interactions, that are not easily represented in traditional networks. Several researchers have applied simplicial complexes to model complex systems effectively \citep{petri2014homological, giusti2016two, sizemore2018importance}.

\subsection{Limitations}

The limitations of both traditional networks and simplicial complexes become evident when examining systems with complex hierarchical organization and nested compositional structures:

\begin{itemize}[leftmargin=*]
  \item \textbf{Social consensus formation}, where opinions cascade through multiple layers of influence \citep{watts1998collective}
  \item \textbf{Biological signaling networks}, where protein complexes assemble and disassemble dynamically \citep{strogatz2001exploring, giovannoni2017dynamic}
  \item \textbf{Financial contagion}, where interactions between institutions have directionality and changing compositions \citep{farmer2009economy}
  \item \textbf{Neural synchronization}, where coordinated firing patterns involve cascades across hierarchical scales \citep{bar2008dynamics, linde2021operad}
\end{itemize}

This paper presents a novel framework that progresses from networks through simplicial complexes to operadic models for analyzing phase transitions and emergent phenomena in complex systems. We demonstrate how changes in operadic structure—including compositional rearrangements, higher-dimensional directed transitions, and operadic persistence features—provide more sensitive and informative indicators of phase transitions than previous methods like networks can offer.
