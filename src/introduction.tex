% Introduction section
\lettrine[lines=2, findent=3pt, nindent=0pt]{C}{omplex systems} are characterized by numerous interacting components that exhibit emergent behaviors and phase transitions not evident from the study of individual components \citep{mitchell2009complexity}. Phase transitions in complex systems are marked by the emergence of sudden, qualitative changes where new collective properties arise from interconnected components.

Network models have been extensively deployed to model and analyze complex systems, capturing pairwise interactions between components \citep{newman2003structure, boccaletti2006complex}. However, networks primarily capture first-order interactions. Higher abstractions of networks like hypergraphs have been employed to capture higher-order interactions \citep{battiston2020networks}. Another popular extension of networks is simplicial complexes, which generalize networks to include higher-dimensional interactions \citep{petri2014homological}. Simplicial complexes have been successfully applied to model complex systems dynamics, capturing higher-order interactions and enabling topological analysis \citep{petri2014homological, giusti2016two, sizemore2018importance}.

However, all of these models have inherent limitations in capturing the complex hierarchical organization and nested compositional structures present in many real-world systems. We therefore propose a novel framework that extends beyond simplicial complexes to operadic models, which provide a more flexible and powerful mathematical structure for representing complex systems dynamics. Operads are algebraic structures that model operations with multiple inputs and a single output, offering a natural framework for capturing compositional and hierarchical processes in complex systems.

The journey toward more powerful and abstract mathematical models for complex systems can be understood as a three-stage progression:

\begin{itemize}[leftmargin=*]
  \item \textbf{Network models} capture only pairwise interactions, providing a first approximation but missing crucial higher-order effects.

  \item \textbf{Simplicial complexes} extend networks to include higher-dimensional interactions, capturing simultaneous multi-agent interactions and enabling topological analysis \citep{battiston2020networks, petri2014homological}.

  \item \textbf{Operads} transcend the limitations of simplicial complexes, offering a richer framework for representing compositional structure, directed higher-order interactions, and dynamic reconfigurations \citep{baez2020network, leinster2004higher, behr2021operad}.
\end{itemize}

The limitations of both traditional networks and simplicial complexes become evident when examining systems with complex hierarchical organization and nested compositional structures:

\begin{itemize}[leftmargin=*]
  \item \textbf{Social consensus formation}, where opinions cascade through multiple layers of influence \citep{watts1998collective}
  \item \textbf{Biological signaling networks}, where protein complexes assemble and disassemble dynamically \citep{strogatz2001exploring, giovannoni2017dynamic}
  \item \textbf{Financial contagion}, where interactions between institutions have directionality and changing compositions \citep{farmer2009economy}
  \item \textbf{Neural synchronization}, where coordinated firing patterns involve cascades across hierarchical scales \citep{bar2008dynamics, linde2021operad}
\end{itemize}

Network models have been widely used to represent complex systems across diverse domains, from social networks to biological systems to technological infrastructures \citep{boccaletti2006complex, newman2003structure}. A network consists of nodes (vertices) connected by edges (links), representing pairwise interactions between entities. The structure of a network can reveal important properties of the system, such as its connectivity, robustness, and information flow \citep{barabasi2004network}. Networks have also been used to capture dynamics of complex systems, including synchronization e.g. in neural networks \citep{arenas2008synchronization}, opinion formation in social networks \citep{castellano2009statistical}, and cascading failures in infrastructure networks \citep{watts2002simple}, percolation theory \citep{cohen2010complex}, and epidemic spreading \citep{pastor2015epidemic}, cascading failures such as in power grids \citep{dobson2007complex}, and many other phenomena.

Hypergraphs are a generalized model of networks which contain hyperedges \citep{berge1984hypergraphs, battiston2020networks}. Hyperedges map a set of source nodes to a set of target nodes, where each edge can span multiple sources and/or targets. Hypergraphs can capture higher-order interactions that are not easily represented in traditional networks \citep{benson2016higher}. They have been used to model complex systems with multi-agent interactions, such as social networks \citep{zhou2007learning}, biological systems \citep{klamt2009hypergraphs}, and technological infrastructures \citep{xu2013hypernetwork}. Examples of such interactions include social consensus formation where collective opinions cascade through multiple layers of influence \citep{neuhäuser2021consensus}, biological signaling networks where protein complexes assemble and disassemble dynamically \citep{ramadan2020hypergraph}, financial contagion where interactions between institutions have directionality and changing compositions \citep{hüser2020financial}, and neural synchronization where coordinated firing patterns involve cascades across hierarchical scales \citep{petri2014homological, giusti2016two, sizemore2018importance}.

Simplicial complexes are a further generalization of networks that include higher-dimensional interactions \citep{petri2014homological, giusti2016two, sizemore2018importance}. A simplicial complex is a collection of 1, 2, 3 ... n higher-dimensional simplices that capture interactions between 1, 2, 3 .. n nodes. For example, a 0-simplex is a node, a 1-simplex is an edge, a 2-simplex is a filled triangle, and a 3-simplex is a solid tetrahedron.  Simplicial complexes can capture complex interactions among multiple entities, such as three-way or four-way interactions, that are not easily represented in traditional networks. Several researchers have applied simplicial complexes to model complex systems effectively \citep{petri2014homological, giusti2016two, sizemore2018importance}.

However, simplicial complexes have inherent limitations: they are rigid, undirected structures that cannot naturally represent composition, hierarchy, or directed multi-level interactions \citep{gong2024higher}. Though there are a few techniques like directed simplicial complexes that can capture directed interactions \citep{masulli2016topology, mukherjee2016random, levi2014directed}, they still lack the flexibility and expressiveness of algebraic structures like operads \citep{baez1997higher, leinster2004higher, kapranov1999operad}.

This paper presents a novel framework that progresses from networks through simplicial complexes to operadic models for analyzing phase transitions and emergent phenomena in complex systems. We demonstrate how changes in operadic structure—including compositional rearrangements, higher-dimensional directed transitions, and operadic persistence features—provide more sensitive and informative indicators of phase transitions than previous methods like networks can offer.
