% Introduction section
\lettrine[lines=2, findent=3pt, nindent=0pt]{C}{omplex systems} are characterized by numerous interacting components that exhibit emergent behaviors and phase transitions not evident from the study of individual components \citep{mitchell2009complexity}. From flocking birds to financial markets, from neural networks to social movements, these systems demonstrate how local interactions cascade across scales to produce system-wide transformations. Understanding such phenomena requires mathematical frameworks that can capture both the multi-scale, hierarchical nature of complex systems and their compositional structure.

The challenge of modeling complex systems has driven the development of increasingly sophisticated mathematical frameworks. Traditional network models capture pairwise interactions but miss higher-order effects. Hypergraphs extend networks to include multi-agent interactions, while simplicial complexes further enable topological analysis through higher-dimensional structures \citep{battiston2020networks}. Yet even these advanced models struggle with a fundamental limitation: they cannot adequately represent systems with complex hierarchical organization and dynamic compositional structures.

Consider social consensus formation, where opinions cascade through multiple levels of influence \citep{watts1998collective}, or biological signaling networks, where protein complexes dynamically assemble to form functional units \citep{giovannoni2017dynamic}. These systems exhibit two key properties identified by \citet{simon1962architecture}: near-decomposability (subsystems interact weakly externally while maintaining strong internal interactions) and scale-invariance (self-similar patterns across organizational levels). Current mathematical frameworks fail to capture how these properties enable phase transitions—those critical moments when local changes reorganize compositional relationships to produce emergent phenomena.

This paper introduces a novel framework based on operads that directly addresses these limitations. Operads, originally developed in algebraic topology, provide a natural language for describing compositional structures and their transformations. We propose T-operads (Type-theoretic operads), which extend operads of wiring diagrams with statistical structure to model how complex systems compose, decompose, and reorganize during phase transitions. This approach offers three key advantages:

\begin{enumerate}[leftmargin=*]
  \item \textbf{Compositional representation}: T-operads explicitly model how subsystems combine to form larger systems while preserving compositional relationships
  \item \textbf{Multi-scale dynamics}: The operadic structure naturally captures interactions across hierarchical levels
  \item \textbf{Phase transition mechanics}: Statistical extensions allow modeling of critical phenomena and emergent behaviors
\end{enumerate}

By bridging category theory, statistical mechanics, and complex systems science, our framework provides new insights into how compositional reorganization drives phase transitions and emergence. We demonstrate applications to diverse systems including neural synchronization, social dynamics, and biological networks, showing how T-operads reveal previously hidden mechanisms underlying complex system behaviors.