% Methodology section
Modeling temporality is central to be able to capture the true essence of complex systems. Components of complex systems interact with one another and evolve across multiple time scales, leading to emergent phenomena that cannot be understood through static or purely structural models. However for modeling such systems, it is sufficient to model the participant components and their interactions as stochastic processes that evolve over time, rather than modeling time explicitly. This implies our model must capture sequences that represent the evolution of states and interactions over time, while preserving causal relationships. The different scales of time are captured by the hierarchical structure of operads, where different levels of the hierarchy can represent dynamics occurring at different time scales, with higher levels operating at slower time scales than lower ones.

We develop a novel mathematical framework, based on WD-operads, to model complex systems through temporal-causal relationships. Our approach enriches the established theory of WD-operads with temporal probability structures, creating operads that naturally capture sequential dynamics and causal interactions.

\subsection{Temporal Probability Spaces}

Let us first define the notion of temporal probability spaces, which will serve as an enriching category for our operads. We denote the category of temporal probability spaces as $\mathbf{TempProb}$.

Each object of $\mathbf{TempProb}$ is a temporal probability space, $\Omega_{\mathcal{W}}$, where each object of the space represents a state of the system at a given time. Hence for each box of the WD-operad $\mathcal{W}$, the space $\Omega_{\mathcal{W}}$ contains the set of all possible execution paths or histories of the system component represented by that box. Additionally, to enable us to assign probabilities to subsets of these execution paths, we equip each temporal probability space with a sigma-algebra $\mathcal{F}$ of measurable events and a probability measure $\mathbb{P}$ that assigns probabilities to these events.

Morphisms in $\mathbf{TempProb}$ are adapted stochastic processes that preserve the temporal structure of the spaces, ensuring that information flows consistently through time. A morphism between two temporal probability spaces represents how the probabilistic evolution of one system component conditions or influences the probability distribution over possible evolutions of another component over time. For example, a morphism $K: \Omega_{\mathcal{W}_1} \to \Omega_{\mathcal{W}_2}$ between two temporal probability spaces can be viewed as a stochastic kernel that maps each execution path in $\Omega_{\mathcal{W}_1}$ to a probability distribution over execution paths in $\Omega_{\mathcal{W}_2}$, while respecting the temporal ordering of events. At time $t$, given history up to time $t$ in $\Omega_{\mathcal{W}_1}$, the morphism $K$ tells us the conditional probability distribution over possible states at time $t$ in $\Omega_{\mathcal{W}_2}$.

Formally, the category $\mathbf{TempProb}$ is defined as follows:

\begin{itemize}
    \item \textbf{Objects}: Temporal probability spaces $(\Omega, \mathcal{F}, \{\mathcal{F}_t\}_{t \geq 0}, \mathbb{P})$ where:
    \begin{itemize}
        \item $\Omega$ is the sample space, representing the set of all possible system trajectories or execution paths.
        \item $\mathcal{F}$ is a sigma-algebra on $\Omega$ - the collection of measurable events (subsets of trajectories) to which we can assign probabilities. Not every arbitrary subset of $\Omega$ is measurable; $\mathcal{F}$ specifies which subsets are "well-behaved" enough for probability theory.
        \item $\{\mathcal{F}_t\}_{t \geq 0}$ is a filtration - an increasing family of sigma-algebras where $\mathcal{F}_s \subseteq \mathcal{F}_t$ for $s \leq t$. This represents the accumulation of information: $\mathcal{F}_t$ captures everything we can know about the system's history up to time $t$.
        \item $\mathbb{P}: \mathcal{F} \to [0,1]$ is a probability measure assigning a likelihood to each measurable event.
    \end{itemize}

    \item \textbf{Morphisms}: Stochastic kernels $K: (\Omega_1, \mathcal{F}_1, \{\mathcal{F}_t^1\}) \to (\Omega_2, \mathcal{F}_2, \{\mathcal{F}_t^2\})$ that are adapted to the filtrations. These represent how one system component probabilistically influences another while respecting causality: at each time $t$, $K$ maps histories in $\Omega_1$ (measurable with respect to $\mathcal{F}_t^1$) to probability distributions over histories in $\Omega_2$. Formally, $K_t$ is $\mathcal{F}_t^1$-measurable for all $t$, ensuring the influence cannot depend on future information.

    \item \textbf{Composition}: For morphisms $K: \Omega_1 \to \Omega_2$ and $L: \Omega_2 \to \Omega_3$, their composition $L \circ K: \Omega_1 \to \Omega_3$ follows the Chapman-Kolmogorov equation, representing the transitive flow of probabilistic influence through chains of system components.

    \item \textbf{Monoidal Structure}: The independent product $(\Omega_1, \mathcal{F}_1, \{\mathcal{F}_t^1\}, \mathbb{P}_1) \otimes (\Omega_2, \mathcal{F}_2, \{\mathcal{F}_t^2\}, \mathbb{P}_2)$ is defined as:
    \begin{itemize}
        \item Sample space: $\Omega_1 \times \Omega_2$ (pairs of independent trajectories)
        \item Sigma-algebra: $\mathcal{F}_1 \otimes \mathcal{F}_2$ (product sigma-algebra)
        \item Filtration: $\mathcal{F}_t^1 \otimes \mathcal{F}_t^2$ at each time $t$
        \item Measure: $\mathbb{P}_1 \times \mathbb{P}_2$ (product measure)
    \end{itemize}
\end{itemize}

This category captures the essential features of complex systems: information accumulates through time (filtration), future states depend probabilistically on past states (stochastic kernels), temporal ordering enforces causality (adaptedness), and independent subsystems can evolve in parallel (monoidal structure).

\subsection{$\sigma$-operads: TempProb-Enriched Wiring Diagrams}

We define $\sigma$-operads as operads enriched in $\mathbf{TempProb}$. Let $\mathcal{W}$ be the standard operad of wiring diagrams with interfaces (finite sets of typed ports) as objects and wiring diagrams as morphisms.

A $\sigma$-operad $\mathcal{W}^{\sigma}$ is the $\mathbf{TempProb}$-enrichment of $\mathcal{W}$, where:
\begin{itemize}
    \item \textbf{Objects}: Same as $\mathcal{W}$ - interfaces representing system boundaries.
    \item \textbf{Hom-objects}: $\mathcal{W}^{\sigma}(X,Y)$ is a temporal probability space rather than a set. Each wiring diagram now carries probabilistic temporal dynamics.
    \item \textbf{Composition}: Operadic composition $\circ$ combined with stochastic composition in $\mathbf{TempProb}$, yielding the temporal evolution of composite systems.
\end{itemize}

\subsubsection{Temporal Causality through Ports}

In $\sigma$-operads, each port carries temporal probabilistic information:
\begin{itemize}
    \item Each port $p$ is associated with an adapted stochastic process $X_p = \{X_p(t)\}_{t \geq 0}$ representing the temporal evolution of information at that interface.
    \item Wires connecting ports encode conditional temporal dependence: if port $p$ connects to port $q$, then $X_q(t)$ depends on the history $\{X_p(s)\}_{s \leq t}$.
    \item The filtration structure ensures causality: information flows from past to future, never backwards in time.
\end{itemize}

\subsubsection{Composition of Temporal Morphisms}

When composing morphisms in $\mathcal{W}^{\sigma}$, we combine both the topological composition of wiring diagrams and the stochastic composition in $\mathbf{TempProb}$:

\begin{enumerate}
    \item \textbf{Topological Composition}: Standard wiring diagram composition connects outputs to inputs through interface matching.
    \item \textbf{Temporal Composition}: The stochastic processes at connected ports are composed using the Chapman-Kolmogorov equation, ensuring that temporal dependencies chain correctly through the composite system.
    \item \textbf{Filtration Compatibility}: The composed system's filtration is the natural extension that preserves the temporal ordering and causal structure.
\end{enumerate}

This composition naturally models how complex systems evolve: subsystems with their own temporal dynamics are connected through their interfaces, creating a larger system whose temporal evolution respects the causal flow through the network structure.

\subsection{Complex Systems as $\sigma$-operad Algebras}

A complex system is modeled as an algebra over a $\sigma$-operad, assigning concrete temporal probability spaces to abstract interfaces and stochastic processes to abstract morphisms. This algebra interprets:
\begin{itemize}
    \item \textbf{Interfaces} as the observational boundaries of system components
    \item \textbf{Wiring diagrams} as the architectural connectivity between components
    \item \textbf{Temporal processes} as the dynamical evolution within and between components
    \item \textbf{Composition} as the emergent dynamics arising from component interactions
\end{itemize}

The key insight is that complex system behavior emerges from the operadic composition structure: local temporal dynamics at individual components compose through the wiring architecture to produce global system behavior, while preserving causality through the temporal ordering inherent in the enriching category.
