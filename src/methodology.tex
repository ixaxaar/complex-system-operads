% Methodology section
In order to model complex systems using $\sigma$-operads, we first look at how to model their structural hallmarks, i.e. near-decomposability including hierarchical structures and modularity followed by self-similarity and scale invariance of the structures.

\subsection{Modeling Near-decomposability}

Modeling near-decomposability requires us to model:

\begin{itemize}
    \item \textbf{Subsystems}: The system can be decomposed into subsystems that interact with each other, and such interactions are specific to the subsystems.
    \item \textbf{Interactions}: The interactions within subsystems are stronger than those between subsystems.
    \item \textbf{Time-scale separation}: Different levels of the hierarchical structure operate at different time scales.
\end{itemize}

\subsubsection{Modeling Subsystems}

A subsystem is a part of a system that can be studied independently, but also interacts with other subsystems. For example:

\begin{itemize}
    \item In a biological system, a cell can be considered a subsystem that interacts with other cells to form tissues.
    \item In a social system, an individual can be considered a subsystem that interacts with other individuals to form groups or communities.
\end{itemize}

Each individual or cell has its own internal dynamics, but also interacts with other individuals or cells in a way that is specific to the subsystem. This means that the interactions within a subsystem are stronger than those between subsystems. For example:

\begin{itemize}
    \item In a social system, an individual may have strong ties to their family or close friends, but weaker ties to acquaintances or strangers.
    \item In a biological system, a cell may have strong interactions with other cells in its tissue, but weaker interactions with cells in other tissues.
\end{itemize}

The members of a subsystem are often more similar to each other than to members of other subsystems, which can lead to the emergence of new properties at different levels of the hierarchy. For example:

\begin{itemize}
    \item In a social system, members of a community may share similar beliefs or behaviors that differ from those of other communities.
    \item In a biological system, cells in a tissue may have similar functions and properties that differ from those of cells in other tissues.
\end{itemize}

If we were to distill the features of a subsystem into features of a $\sigma$-operad, we would have the following mapping:

\begin{itemize}
    \item \textbf{Operations as Subsystems:} Each subsystem corresponds to an operation (a box) in the wiring diagram. Its input and output ports define its interaction interface.
    \item \textbf{Morphisms as Interactions:} The interactions between components within or between subsystems are represented by the morphisms (wires and internal operations) of the $\sigma$-operad.
    \item \textbf{Composition as Hierarchy and Modularity:} The operadic composition (substitution of operations) naturally models hierarchical organization and modularity, allowing smaller $\sigma$-operads to combine into larger ones.
    \item \textbf{Internal Dynamics as Statistical State:} The specific internal dynamics of a subsystem are captured by the stochastic state space attached to its operation, enabling the modeling of probabilistic behaviors and fluctuations.
\end{itemize}
