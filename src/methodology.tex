% Methodology section
We develop a novel mathematical framework to model complex systems that exhibit both structural connectivity and internal statistical dynamics. Our approach extends the traditional operadic framework of wiring diagrams by enriching it with network structures, specifically Bayesian networks, to capture the probabilistic relationships inherent in such systems.

\subsection{Network-Enriched Wiring Diagram Operads}

To model complex systems that exhibit both structural connectivity and internal statistical dynamics, we cannot rely on standard wiring diagram operads alone. We require a framework that enriches the topological structure of wiring diagrams with an additional layer of information — specifically, a network structure that can encode causal and probabilistic dependencies. We propose a tiered construction: first defining a general network-enriched operad, and then instantiating it with Bayesian networks to capture statistical mechanics. A thing to take specific note of is that we posit that modeling causality is all we need to model complex adaptive systems, without the need to explicity model time.

\subsubsection{The Network-Enriched WD-Operad}

Let $\mathcal{W}$ be the operad of wiring diagrams, where objects are interfaces (finite sets of typed ports) and morphisms are wiring diagrams. We define an \textbf{Network-Enriched WD-Operad}, denoted by $\mathcal{E}$, as an extension of $\mathcal{W}$ such that:

A morphism in $\mathcal{E}(X, Y)$ is a pair $(W, G)$, where:
\begin{itemize}
    \item $W \in \mathcal{S}(X, Y)$ is a standard wiring diagram defining the physical or logical connectivity.
    \item $G = (V, E)$ is a graph structure (the "network") whose nodes $V$ are mapped to the ports of $W$.
\end{itemize}

The composition law in $\mathcal{E}$ must respect both the wiring substitution and the graph structure. If we compose $(W_{out}, G_{out}) \circ ((W_{in_1}, G_{in_1}), \dots, (W_{in_k}, G_{in_k}))$, the resulting graph $G_{new}$ is formed by the union of the graphs, identifying nodes that correspond to connected ports (the "gluing" step).

\subsubsection{Compatibility Conditions (Network Enrichment)}

For the enrichment to be meaningful, the graph $G$ must be compatible with the wiring diagram $W$. We enforce the following \textit{compatibility conditions}:

\begin{enumerate}
    \item \textbf{Node Mapping}: There exists a bijection between the set of vertices $V(G)$ and the set of ports in $W$.
    \item \textbf{Edge Consistency}: If there is a wire in $W$ connecting port $u$ to port $v$, the graph $G$ must allow for a relationship between the corresponding nodes (e.g., a directed edge $u \to v$ if $G$ is directed).
    \item \textbf{Acyclicity}: Since wiring diagrams represent feed-forward composition, the union of graph structures during composition must not introduce cycles if the underlying system is to remain causal.
\end{enumerate}

\subsubsection{Symmetry}

The symmetry actions in $\mathcal{E}$ extend those in $\mathcal{S}$. A permutation $\sigma \in \Sigma_k$ acting on the input interfaces of a morphism $(W, G)$ induces a relabeling of the corresponding nodes in $G$. The enriched operad must satisfy the condition that the statistical or causal semantics encoded by $G$ remain invariant under such permutations, ensuring that the operadic structure respects the symmetry of input ordering.

\subsubsection{Algebra over Network-Enriched Operads}

To give concrete meaning to our Statistical Wiring Diagrams, we must define an \textit{algebra} over the $\mathcal{SWD}$ operad. In operad theory, an algebra is a map that assigns semantic objects to the abstract types and concrete operations to the abstract morphisms, preserving the composition structure.

Formally, an algebra $\mathcal{A}$ over the operad $\mathcal{SWD}$ is a strict symmetric monoidal functor from the category underlying $\mathcal{SWD}$ to a target symmetric monoidal category of semantics, typically the category of Markov kernels, denoted $\mathbf{Mark}$.

\begin{itemize}

    \item \textbf{Objects:} For every type $t$ in our interface set, the algebra assigns a measurable space $\mathcal{A}(t) = (X_t, \Sigma_t)$. A composite interface $I = \{t_1, \dots, t_k\}$ is mapped to the product space $\mathcal{A}(I) = \proW_{i} X_{t_i}$.

    \item \textbf{Morphisms:} For every morphism $F = (W, \mathcal{B})$ in $\mathcal{SWD}(I, O)$, the algebra assigns a Markov kernel:

    \[ \mathcal{A}(F): \mathcal{A}(I) \to \mathcal{A}(O) \]

    This kernel $K(x, dy)$ represents the conditional probability distribution $P(\text{outputs} \mid \text{inputs})$ encoded by the Bayesian network $\mathcal{B}$.

\end{itemize}



The defining feature of this algebra is the preservation of composition. If $F = G \circ (H_1, \dots, H_k)$ is a composite morphism in $\mathcal{SWD}$, then the corresponding kernel must be the composition of the constituent kernels:

\[ \mathcal{A}(F) = \mathcal{A}(G) \circ (\mathcal{A}(H_1) \otimes \dots \otimes \mathcal{A}(H_k)) \]

In the category $\mathbf{Mark}$, this composition corresponds to the Chapman-Kolmogorov equation (integration over intermediate states). Thus, the abstract "gluing" of Bayesian networks in the operad perfectly mirrors the chaining of conditional probabilities in the semantic domain. This guarantees that our topological construction of complex systems yields valid, computable statistical models.

\subsection{Bayesian Networks as a Candidate Algebra}

Having defined the general structure of $\mathcal{E}$, we now select a specific candidate for the network layer to model statistical mechanics: \textbf{Bayesian Networks}.

We define the \textit{Statistical Wiring Diagram (SWD) Operad} as the specific instance of $\mathcal{E}$ where the enriching graphs are Bayesian Networks equipped with conditional probability distributions.

\subsubsection{Formal Derivation}

A morphism in the SWD operad is a pair $(W, \mathcal{B})$, derived as follows:

\begin{enumerate}
    \item \textbf{Topological Base}: We start with the wiring diagram $W$.
    \item \textbf{Probabilistic Enrichment}: The graph $G$ becomes the Directed Acyclic Graph (DAG) of a Bayesian Network $\mathcal{B}$.
        \begin{itemize}
            \item \textbf{Nodes}: Each port $p$ in $W$ is associated with a random variable $X_p$.
            \item \textbf{Edges}: Directed edges in $\mathcal{B}$ represent conditional dependencies.
            \item \textbf{Parameters}: Each node $X_p$ is equipped with a Conditional Probability Distribution (CPD) $P(X_p | \text{Pa}(X_p))$, where $\text{Pa}(X_p)$ are the parents of $X_p$ in the graph.
        \end{itemize}
\end{enumerate}

\subsubsection{Composition of Statistical Morphisms}

The composition of two statistical morphisms $(W_2, \mathcal{B}_2) \circ (W_1, \mathcal{B}_1)$ follows the logic of the enriched operad but adds the probabilistic algebra:

\begin{itemize}
    \item \textbf{Graph Gluing}: We form the union of the DAGs. If an output port of $W_1$ connects to an input port of $W_2$, the corresponding random variables are identified ($X_{out} \equiv X_{in}$), fusing the networks.
    \item \textbf{Factorization}: The joint probability distribution of the composed system is the product of the individual factors. This preserves the Markov property: the global system's probability distribution factorizes over the structure of the composed wiring diagram.
\end{itemize}

This construction proves that SWDs are not an ad-hoc definition but a rigorous algebra over the category of network-enriched wiring diagrams.

\subsubsection{Symmetry and Exchangeability}

A critical feature of operads is their symmetry—the ability to permute the order of inputs without changing the operation's outcome (up to isomorphism). For $\mathcal{SWD}$, this imposes a strict consistency condition on the statistical structure.

Let $\sigma \in \Sigma_k$ be a permutation of the $k$ input interfaces of a morphism $F$. The action of $\sigma$ on the wiring diagram $W_F$ is topological: it physically swaps the input bundles. For the associated Bayesian network $\mathcal{B}_F$, this action corresponds to a relabeling of the random variables associated with the input ports.

The **Symmetry Condition** for SWD operads requires that the probabilistic semantics are invariant under this relabeling. Formally, if $\mathbf{X}_{in} = (X_1, \dots, X_k)$ are the input variable sets, then for any permutation $\sigma$:
\[ P_F(\text{outputs} \mid X_{\sigma(1)}, \dots, X_{\sigma(k)}) \cong P_F(\text{outputs} \mid X_1, \dots, X_k) \]
where $\cong$ denotes equality of distributions under the coordinate transformation induced by $\sigma$. This ensures that the statistical mechanics of the system depends only on the causal connectivity (the wiring), not on the arbitrary labeling of the input ports. In statistical terms, this is a form of \textit{exchangeability} conditional on the wiring structure.
