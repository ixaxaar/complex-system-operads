% methodology
In order to model complex systems using T-operads, we first look at how to model their structural hallmarks, i.e. near-decomposability including hierarchcal structures and modularity followed by self-similarity and scale invariance of the structures.

\subsection{Modeling Near-decomposability}

Modeling neardecomposability requires us to model:

\begin{itemize}
    \item \textbf{Subsystems}: The system can be decomposed into subsystems that interact with each other, and such interactions are specific to the subsystems.
    \item \textbf{Interactions}: The interactions within subsystems are stronger than those between subsystems.
    \item \textbf{Time-scale separation}: Different levels of the hierarchical structure operate at different time scales.
\end{itemize}

\subsubsection{Modeling Subsystems}

A subsystem is a part of a system that can be studied independently, but also interacts with other subsystems. For example, in a biological system, a cell can be considered a subsystem that interacts with other cells to form tissues. In a social system, an individual can be considered a subsystem that interacts with other individuals to form groups or communities. Each individual or cell has its own internal dynamics, but also interacts with other individuals or cells in a way that is specific to the subsystem. This means that the interactions within a subsystem are stronger than those between subsystems. For example, in a social system, an individual may have strong ties to their family or close friends, but weaker ties to acquaintances or strangers. Similarly, in a biological system, a cell may have strong interactions with other cells in its tissue, but weaker interactions with cells in other tissues. The members of a subsystem are often more similar to each other than to members of other subsystems, which can lead to the emergence of new properties at different levels of the hierarchy. For example, in a biological system, cells in a tissue may have similar functions and properties, while cells in different tissues may have different functions and properties. In a social system, individuals in a community may share similar beliefs or behaviors, while individuals in different communities may have different beliefs or behaviors.

If we were to distill the features of a subsystem into features of a T-operad, we would have the following:

\begin{itemize}
    \item \textbf{Compositional hierarchy}:

\begin{itemize}
    \item \textbf{Compositional structure}: The subsystem can be represented as a T-operad, where the objects are the members of the subsystem and the morphisms are the interactions between them.
    \item \textbf{Hierarchical structure}: The T-operad can be decomposed into smaller T-operads, each representing a subsystem. This allows us to model the hierarchical structure of the system.
    \item \textbf{Modularity}: The T-operad can be composed with other T-operads to form larger T-operads, allowing us to model the modularity of the system.
\end{itemize}

