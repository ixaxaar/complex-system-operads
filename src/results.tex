
The $\sigma$-operad framework naturally captures fundamental complex systems properties through its mathematical structure. We demonstrate how key features identified in complex systems literature emerge directly from operadic composition.

\subsection{Built-in support for structural properties}

The operadic structure provides native support for fundamental characteristics of complex systems.

\subsubsection{Structural: hierarchy, modularity and near-decomposability}

The $\sigma$-operad framework inherently captures the fundamental structural properties that define complex systems. \textbf{Near-decomposability} --- systems with stronger internal than external interactions --- emerges naturally from the operadic structure where each operation (box) defines a clear subsystem with well-defined input/output interfaces that enforce boundary conditions through composition rules. Strong internal interactions are captured by the temporal probability space within each operation, while weak external interactions are modeled by stochastic kernels between operations, creating the characteristic \textbf{interaction strength separation}. The framework supports \textbf{hierarchical organization} through nested composition where smaller operations combine into larger ones, with each level of composition corresponding to a distinct hierarchical level where higher-level operations operate on outputs of lower-level ones. Different hierarchical levels operate at different temporal scales through the filtration structure $\{\mathcal{F}_t\}$, allowing higher-level operations to have coarser temporal discretization than lower-level ones, naturally capturing \textbf{multi-scale dynamics}. This \textbf{time-scale separation} is enforced by the 2-step composition process that first forms independent products then applies dependencies, ensuring that subsystem dynamics remain distinguishable across scales.

\subsubsection{Modularity and compositionality}

Complex systems exhibit \textbf{modular organization} where components can be recombined in different configurations, a property that emerges directly from operadic composition principles. Each operation functions as a module with standardized interfaces (ports) that enable systematic construction of larger systems from smaller components through \textbf{operadic substitution}. The composition rules ensure that modules can only be connected when their interfaces match, preventing incompatible combinations while enabling flexible recombination of components. This \textbf{interface compatibility} creates a \textbf{compositional semantics} where the meaning (temporal dynamics) of a composite system is determined by the meanings of its components and their interaction patterns, following strict compositional principles that guarantee predictable system behavior from modular assembly.

\subsubsection{Scale-invariance and fractal structure}

\textbf{Scale-invariance} and \textbf{self-similar patterns} across scales emerge naturally from operadic self-composition, enabling the framework to capture the fractal-like organization characteristic of many complex systems. Operations can compose with themselves at different scales through \textbf{recursive application} $f \circ f \circ f \cdots$, creating hierarchical structures where the same interaction pattern repeats at multiple organizational levels. This self-similar composition generates the \textbf{recursive structures} observed in biological branching patterns, social network hierarchies, and economic market dynamics. The temporal probability spaces can exhibit \textbf{scale-free behavior} when stochastic kernels preserve \textbf{power-law distributions} during composition—if the initial distribution follows $\mathbb{P}(X) \sim X^{-\alpha}$, appropriately chosen kernels maintain this scaling relationship through successive compositions. This mathematical property enables the framework to model systems that lack \textbf{characteristic scales}, such as neural networks with scale-free connectivity or ecosystems with power-law species distributions. The operadic structure naturally supports \textbf{recursive definitions} where complex systems are defined in terms of smaller versions of themselves, generating the self-similar hierarchies that exhibit \textbf{statistical invariance} across observation scales.
